\documentclass[12pt, preprint]{aastex}

\newcommand{\project}[1]{\textsl{#1}}
\newcommand{\Kepler}{\project{Kepler}}
\newcommand{\name}{CPM}

\begin{document}

\title{AAS Abstract}

The precision of \Kepler\ photometry for exoplanet science---the most
precise photometric measurements of stars ever made---appears to be
limited by unknown or untracked variations in spacecraft pointing and
temperature, and unmodeled stellar variability.
To exploit all the information of the \Kepler\ data, here we present \name, 
a pixel level data-driven model intended to model all these variations(or really of their impact on the photometry) and preserve transit signals.
Importantly (and uniquely),
  \name\ works at the pixel level (not the photometric measurement level);
  it can capture more fine-grained information about the variation of the spacecraft
  (especially regarding pointing and point-spread function)
  than is available in the pixel-summed photometry. 
With the help of L2-regularization, \name\ predicts the target pixel value by linearly combining large number of pixels from other stars that share the imstrument vaiabilities. 
To preserve the transit signals, we adopt the train-and-test framework, with which information of the transit will be perfectly isolated from the model. 
We optimize the model by cross-validating over four hyper-parameters---number of predictor stars, strength of L2-regularization, size of auto-regressive compoents and the strength of auto-regressive L2-regularization.
After examining numbers of target stars, we make a general set of hyper-parameters that works quite well on most of the stars with Kepler-band magnitude around 11-12. 
To show the advantages of \name,  a set of target stars with planet-transits are choosed, \name\ can always outperform the \Kepler\ Presearch Data Conditioning method,
with higher S/N(signal to noise ratio).

\end{document}
