% This file is part of the causal-kepler project
% Copyright 2013 the authors.

\documentclass[12pt, preprint]{aastex}

\newcommand{\notenglish}[1]{\textit{#1}}
\newcommand{\sic}{\notenglish{sic}}
\newcommand{\project}[1]{\textsl{#1}}
\newcommand{\Kepler}{\project{Kepler}}
\newcommand{\name}{PLM}

\begin{document}

\title{%
  A data-driven, pixel-level model to improve the precision of \Kepler\ photometry%
}
\author{%
  Dan~Foreman-Mackey\altaffilmark{\ref{CCPP}},
  David~W.~Hogg\altaffilmark{\ref{CCPP},\ref{MPIA},\ref{email}},
  Rob~Fergus\altaffilmark{\ref{Courant}},
  Stefan Harmeling\altaffilmark{\ref{MPIIS}},
  Bernhard~Sch\"olkopf\altaffilmark{\ref{MPIIS}},
  others%
}

\newcounter{address}
\setcounter{address}{1}
\altaffiltext{\theaddress}{\stepcounter{address}\label{CCPP}%
  Center for Cosmology and Particle Physics, Department of Physics, New York University}
\altaffiltext{\theaddress}{\stepcounter{address}\label{MPIA}%
  Max-Planck-Institut f\"ur Astronomie, Heidelberg, Germany}
\altaffiltext{\theaddress}{\stepcounter{address}\label{email}%
  To whom correspondence should be addressed; \texttt{<david.hogg@nyu.edu>}.}
\altaffiltext{\theaddress}{\stepcounter{address}\label{MPIIS}%
  Max-Planck-Institut f\"ur Intelligente Systeme, T\"ubingen}
\altaffiltext{\theaddress}{\stepcounter{address}\label{Courant}%
  Courant Institute of Mathematical Sciences, New York University}
%% \altaffiltext{\theaddress}{\stepcounter{address}\label{Oxford}%
%%   Department of Physics, Oxford University}
%% \altaffiltext{\theaddress}{\stepcounter{address}\label{Ames}%
%%   NASA Ames Research Center}
%% \altaffiltext{\theaddress}{\stepcounter{address}\label{CfA}%
%%   Harvard--Smithsonian Center for Astrophysics}
%% \altaffiltext{\theaddress}{\stepcounter{address}\label{UCL}%
%%   Department of Physics and Astronomy, University College London}
%% \altaffiltext{\theaddress}{\stepcounter{address}\label{CMU}%
%%   McWilliams Center for Cosmology, Carnegie Mellon University}
%% \altaffiltext{\theaddress}{\stepcounter{address}\label{Caltech}%
%%   Department of Astronomy, California Institute of Technology}
%% \altaffiltext{\theaddress}{\stepcounter{address}\label{Columbia}%
%%   Department of Astronomy, Columbia University}

\begin{abstract}
The precision of \Kepler\ photometry%
  ---the most precise photometric measurements of stars ever made---%
  appears to be limited by unknown or untracked variations
  in spacecraft pointing, point-spread function, stellar motions and parallaxes, and temperature.
Here we present \name,
  a data-driven model of these variations in \Kepler\ data (or really of their impact on the photometry).
Importantly (and uniquely),
  \name\ works at the pixel level (not the photometric measurement level);
  it can capture more fine-grained information about the variation of the spacecraft
  (especially regarding pointing and point-spread function)
  than is available in the pixel-summed photometry.
\name\ is extreme in that it has enormous flexibility and it only uses data (and meta-data) to model data:
\name\ provides a prediction for each readout pixel built from a linear combination of the readouts
  from very large numbers of nearby pixels (plus some spacecraft information);
  the choice of nearby pixels is guided by ideas from causal inference.
\name\ avoids over-fitting by employing a train-and-test formalism
  designed to ensure that transit-like photometric events and short-timescale stellar variability
  cannot be captured or distorted by the model;
  it is designed to remove spacecraft-induced variability but not intrinsic stellar variability.
We show that \name\ outperforms the \Kepler\ Presearch (\sic) Data Conditioning method on a set of example stars.
We release open-source code that provides \name\ output for any star in the existing \Kepler\ Archive,
  and we discuss applications for other missions,
  including any possible \Kepler\ two-wheel mission.
\end{abstract}

\section{Ultra-precise photometry}

The photometric measurements of stars made by the \Kepler\ Satellite are precise enough
  to permit discovery of exoplanet transits with depths smaller than $10^{-4}$.
This precision results from great spacecraft stability,
  supplemented by various methods for removing small residual spacecraft-induced and stellar-variability trends in the brightnesses,
  either filtering the data (with things like median filters; CITE)
  or fitting the data with flexible models (like polynomials or splines or Gaussian Processes; CITE).
When employed in the service of exoplanet search and characterization,
  these methods are usually agnostic about whether photometric variations originate in the spacecraft or in the star itself;
  that is, they obliterate intrinsic stellar variability along with spacecraft issues.

In general, there are many reasons for apparent photometric variability in a \Kepler\ source.
There is intrinsic stellar variability,
  which is of interest to some and a nuisance to others.
There is also variability of overlapping fainter stars;
  that is, confusion noise combined with variability of the confusing sources.
There are small changes in spacecraft pointing,
  which leads to slightly different illumination of the focal-plane pixels,
  and thus different sensitivity to errors (problems) in the device flat-field or sensitivity map.
There are also \emph{intra-pixel} sensitivity variations that can contribute (CITE WHITEPAPER, SPITZER).
There are small changes in spacecraft temperature,
  which lead to point-spread function (PSF) and differential (across the focal plane) pointing changes.
These also lead to changes in pixel and intra-pixel illumination.
Stellar proper motion and parallaxes must do the same, at some level.
There is electronic cross-talk between detectors and charge-transfer inefficiency,
  that can effectively transfer variability from one source to another.
There are also possible changes to the detector sensitivity over time,
  and possibly sources of variability not yet considered.
The remarkable thing about \Kepler\ is that it is trying to measure stars at a level of precision
  much higher than ever previously attempted;
  new effects really \emph{must} appear at some point.
In \figurename~HOGG, we show the pixel-level variability in the \Kepler\ data
  near one bright, non-variable star.

We propose and advise dealing with these variations in \Kepler\ light-curves by \emph{modeling} them.
This model can be a physical model (of the spacecraft PSF, pointing, temperature, and so on)
  or it can be a flexible, effective model that has no direct interpretation in terms of physical spacecraft parameters.
In principle a physical model will do a better job,
  because it necessarily embodies more prior information,
  but it requires research and intuition about dominant effects,
  and this research and intuition is often wrong or incomplete.
The model we propose here is in the non-physical, effective category.
The \Kepler\ community is familiar with these kinds of models;
  to our knowledge, \emph{all} successful light-curve ``de-trending'' methods
  are flexible, effective models.

One such method---one that is designed to model spacecraft-induced problems
  but \emph{not} interfere with measurements of stellar variability---%
  is the \Kepler\ Presearch (\sic) Data Conditioning (PDC) method (CITE).
The PDC ``de-trends'' the \Kepler\ lightcurves by fitting them with a small set of basis lightcurves
  generated from a principal components analysis (PCA) of filtered lightcurves.
That is, it uses data to model data,
  regularizing the fit (and avoiding over-fitting) by filtering and restricting the dimensionality (through PCA).
The method proposed here, \name, is very similar in spirit to the PDC.
The main differences are
  (1)~that \name\ works in the pixel domain, not the lightcurve domain, so it has access to more fine-grained information,
  (2)~that \name\ has far more freedom (far more parameters) than the PDC
  but it strictly avoids over-fitting the lightcurves on exoplanet-transit time-scales
  through strong regularization and a train-and-test framework, and
  (3)~that when \name\ fits a star, there is no possibility of any contribution of the star itself to the fit basis vectors.
The two methods (\name\ and PDC) are similar however,
  in that they both make the assumption that whatever spacecraft effects are imprinting variability on a stellar lightcurve
  must also be imprinting similar or related variability on other lightcurves.
By going to the pixel level, \name\ makes it easier for the model to capture variability
  that is coming through variations in the centroids and point-spread function
  from spacecraft pointing, roll, and temperature.

Before we start, a few reminders about the \Kepler\ data are in order:
The satellite always observes precisely the same field, at fixed pointing (as closely as possible).
The satellite is rolled by 90~deg every 90-ish days (for Solar Angle constraints).
The PSF varies strongly across the field and is badly sampled.
The stars span a huge range in brightness;
  some of \Kepler's most important targets even saturate the device and bleed charge.
The stellar photometry returned by the \Kepler\ SAP and PDC pipelines is based on
  straight, \emph{unweighted} sums of pixels in small patches centered on the stellar centroid.
We will return to this latter point at the end;
  this kind of photometry cannot be optimal;
  it must be possible to do a better job with the photometric measurements.
That's beyond the scope of this paper but a place for a valuable intervention on the \Kepler\ data.

The scope of this paper is an intervention---\name---that makes the \Kepler\ photometry more precise.
We also provide an interface to the \Kepler\ data that delivers improved ``\name\ photometry'' for every \Kepler\ target,
  and all the relevant code in public, open-source repositories.

\section{Generalities: Causality and data-driven models}

\section{Specifics: PLM and it's hyper-parameters}

\section{Examples and results}

\section{Discussion}

HOGG: Return to the SAP photometry ``unweighted sum'' issue here.

\acknowledgements
It is a pleasure to thank the whole \Kepler\ Team
  for designing, delivering, and operating a great facility,
  and for making all of the data public, in all its rawest forms, through the MAST interface.
We are also pleased to thank
  Ruth~Angus (Oxford),
  Bekki~Dawson (Berkeley),
  Michael~Hirsch (UCL),
  Dustin~Lang (CMU),
  Benjamin~T.~Montet (Caltech),
  and
  David~Schiminovich (Columbia)
for valuable discussions, input, encouragement, and advice.

WHITEPAPER AUTHORS:  FEEL FREE TO CONTRIBUTE TO THIS PAPER AND MOVE FROM THE ACK LIST TO THE AUTHOR LIST.

\end{document}
